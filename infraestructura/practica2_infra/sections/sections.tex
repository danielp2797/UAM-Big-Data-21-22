\section*{Aclaraciones}
\addcontentsline{toc}{section}{Aclaraciones}
Para realizar la práctica se ha utilizado la traza proporcionada en el curso. En la sección de ejercicios se presenta la idea detrás de cada consulta, el código ejecutado y una tabla que recoge las primeras filas del resultado.
\section*{Ejercicios}
\addcontentsline{toc}{section}{Ejercicios}
\textbf{Ejercicio 1. Crear un script HIVE que obtenga la serie temporal en bits/s con granularidad 1 segundo para la traza de análisis. (0,75 puntos)}\\\\
Para obtener la serie se agregan todos los paquetes por timestamp y se suma la longitud multiplicada por ocho para obtener el dato en bits.
\begin{lstlisting}[caption=Consulta para obtener la serie]
SELECT ts, sum(len)*8 as bits
from pcaps 
group by ts 
order by ts;
\end{lstlisting}
En la tabla \ref{tab:c1} se muestran unas líneas del resultado.
\begin{table}[ht]
	\centering
	% To place a caption above a table
	\caption{Ejemplo del resultado de la consulta}
	\begin{tabular}[t]{cc}
		timestamp & bits \\
		\hline
		1511781613 & 1216 \\
		1511781621 & 1280 \\
		1511781622 & 1616 \\
		1511781630 &1648 \\
		1511781635 & NULL \\
		1511781643 & 1184 \\
		1511781651 & 2752 \\
		1511781654 & 1104 \\
	\end{tabular}
	% Or to place a caption below a table
	\label{tab:c1}
\end{table}%
\\
Se observa que en algunos segundos del timestamp se ha producido algún error ya que hay valores NULL en la columna bits. Además, la serie no es continua ya que no estan todos los segundos en las filas. \\\\
\textbf{Ejercicio 2. Crear un script HIVE que obtenga registros de flujos junto con el número de bytes y paquetes
	para la traza de análisis. Los registros de flujos han de contener: Dirección IP origen y destino, Puerto origen y destino, Protocolo (TCP o UDP), Número de bytes, Número de paquetes. (0,75 puntos)}\\\\
Para obtener el registro de flujos se agregan las observaciones por origen, destino y protocolo para despues calcular la cantidad de bytes y paquetes. 
\begin{lstlisting}[caption=Consulta para obtener el registro de flujos]
SELECT src,
	dst, 
	protocol,
	sum(len) AS bytes,
	count(*) AS packages 
FROM default.pcaps
WHERE protocol IN ('UDP', 'TCP')
GROUP BY src, dst, protocol;
\end{lstlisting}
En la tabla \ref{tab:c2} se observa un ejemplo del registro.
\begin{table}[ht]
	\centering
	% To place a caption above a table
	\caption{Ejemplo del registro}
	\begin{tabular}[t]{ccccc}
		src	& dst &	protocol &	bytes &	packages\\
		\hline
		8.254.173.126 & 192.168.158.128	& TCP &	63212058 &	22610 \\
			192.168.158.128 &	8.254.173.126 &	TCP &	434872 &	18960\\
			130.206.192.17 &	192.168.158.128 &	TCP  &	16135936 &	7766 \\
			192.168.158.128 &	192.168.158.2 &	UDP &	254928 &	6988\\
			192.168.158.2 &	192.168.158.128 &	UDP &	840238 &	6940
	\end{tabular}
	% Or to place a caption below a table
	\label{tab:c2}
\end{table}%
\\
\textbf{Ejercicio 3. Crear un script HIVE que obtenga los ‘Host’ (a nivel HTTP) más populares (en número de
	peticiones) en la traza de análisis. (0.75 puntos)}\\\\
En este ejercicio se precisa ampliar la tabla provista en el curso con el campo \textit{header\_host}. Una vez añadido, basta hacer un recuento de peticiones (\textit{http\_request}) agregando por los valores de host y ordenar descendentemente para obtener el ranking de hosts más populares.
\begin{lstlisting}[caption=Consulta para obtener el ranking de hosts]
ADD JAR hdfs:///user/uambdXX/libs/
hadoop-pcap-serde-1.2-
SNAPSHOT-jar-with-dependencies.jar;

SET net.ripe.hadoop.pcap.io.reader.class=
net.ripe.hadoop.pcap.HttpPcapReader;
SELECT header-host, count(http-request) as count
	FROM http 
	GROUP BY header-host 
	ORDER BY rank DESC LIMIT 10;
\end{lstlisting}
En la tabla \ref{tab:c3} se muestra un ejemplo del resultado.
\begin{table}[ht]
	\centering
	% To place a caption above a table
	\caption{Ejemplo del resultado de la consulta}
	\begin{tabular}[t]{cc}
		host & count \\
		\hline
		fastlane.rubiconproject.com	& 100 \\
		ocsp.comodoca.com &	72\\
		ib.adnxs.com &	56\\
		ocsp.godaddy.com &	48\\
		secure.adnxs.com &	46\\
		ams1-ib.adnxs.com &	30\\
		beacon.krxd.net &	30\\
		tribune-d.openx.net &	28\\
		img.rtve.es	& 24 
	\end{tabular}
	% Or to place a caption below a table
	\label{tab:c3}
\end{table}%
\\
\textbf{Ejercicio 4. Crear un script HIVE que obtenga la cantidad de ‘UserAgent’ distintos junto con el número de
	peticiones observadas en la traza de análisis. (0.75 puntos)}\\\\
En este caso, tras añadir el campo \textit{header\_user-agent}, se hace un recuento de la cantidad de agentes distintos junto al recuento de todas las filas.
\begin{lstlisting}[caption=Consulta para obtener el resúmen de agentes.]
ADD JAR hdfs:///user/uambdXX/libs/
hadoop-pcap-serde-1.2-
SNAPSHOT-jar-with-dependencies.jar;

SET net.ripe.hadoop.pcap.io.reader.class=
net.ripe.hadoop.pcap.HttpPcapReader;

SELECT COUNT(DISTINCT header-user-agent) AS uacount,
COUNT(*) AS trcount 
FROM http;
\end{lstlisting}
En la tabla \ref{tab:c4} se muestra el resultado.\\
\begin{table}[ht]
	\centering
	% To place a caption above a table
	\caption{Ejemplo del resultado de la consulta}
	\begin{tabular}[t]{cc}
		ua\_count & tr\_count \\
		\hline
		 2 &	334698
	\end{tabular}
	% Or to place a caption below a table
	\label{tab:c4}
\end{table}%
\\
\textbf{Ejercicio 5. Crear un script HIVE que obtenga las diferentes líneas de respuesta HTTP junto con el número de veces que se han observado. (0,75 puntos)}\\\\
Para resolver este ejercicio se hace un recuento de observaciones agregado por la respuesta http.
\begin{lstlisting}[caption=Consulta para obtener el recuento de respuestas]
ADD JAR hdfs:///user/uambdXX/libs/
hadoop-pcap-serde-1.2-
SNAPSHOT-jar-with-dependencies.jar;

SET net.ripe.hadoop.pcap.io.reader.class=
net.ripe.hadoop.pcap.HttpPcapReader;

SELECT http_response,
	 count(*) AS res_count
	 FROM http
	 GROUP BY http_response;
\end{lstlisting}
En la tabla \ref{tab:c5} se muestran las primeras líneas del resultado.

\begin{table}[ht]
	\centering
	% To place a caption above a table
	\caption{Ejemplo del resultado de la consulta}
	\begin{tabular}[t]{cc}
		http\_response & res\_count \\
		\hline
		NULL &	332400\\
		HTTP/1.1 200 OK&	1792\\
		HTTP/1.1 302 Found&	250\\
		HTTP/1.1 302 Moved Temporarily&	96\\
		HTTP/1.1 204 No Content&	60\\
	\end{tabular}
	% Or to place a caption below a table
	\label{tab:c5}
\end{table}%
Se observa un número de respuestas OK superior al resto, un resultado razonable si tenemos en cuenta que la traza se genera mediante un uso normal de internet.\\\\
\textbf{Ejercicio 6. Crear un script HIVE que obtenga la popularidad de los nombre de dominio (por número de
	peticiones) en la traza de análisis. (0,75 puntos)}\\\\
Para resolver este ejercicio nos basamos en la cantidad de requests DNS. Al ser un protocolo de resolución de direcciones, para encontrar los dominios más populares basta contar la cantidad de veces que se ha resuelto dicho dominio por DNS.\\\\
En esta consulta los nombres de dominio se obtienen del primer substring (separndo por espacios) de la columna \textit{dns\_request} y se hace un recuento agregando por esta columna calculada. No se ha encontrado una forma más sencilla y directa de hacerlo con el serde proporcionado.
\begin{lstlisting}[caption=Consulta para obtener el ranking de dominios]
	ADD JAR hdfs:///user/uambdXX/libs/
	hadoop-pcap-serde-1.2-
	SNAPSHOT-jar-with-dependencies.jar;
	
	SET net.ripe.hadoop.pcap.io.reader.class=
	net.ripe.hadoop.pcap.HttpPcapReader;
	
SELECT header_host, count(*) AS counter FROM http GROUP BY header_host ORDER BY counter DESC LIMIT 10;
\end{lstlisting}
En la tabla \ref{tab:c5} se muestran las primeras líneas del resultado.

\begin{table}[ht]
	\centering
	% To place a caption above a table
	\caption{Ejemplo del resultado de la consulta}
	\begin{tabular}[t]{cc}
		http\_response & res\_count \\
		\hline
		NULL &	332400\\
		HTTP/1.1 200 OK&	1792\\
		HTTP/1.1 302 Found&	250\\
		HTTP/1.1 302 Moved Temporarily&	96\\
		HTTP/1.1 204 No Content&	60\\
	\end{tabular}
	% Or to place a caption below a table
	\label{tab:c5}
\end{table}%
